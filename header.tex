
\usefonttheme[onlymath]{serif}
% use article-like math letters, from http://tex.stackexchange.com/questions/34265/how-to-get-beamer-math-to-look-like-article-math


% linear model equations
\newcommand\LMi{\mathrm{(LM1)}}
\newcommand\LMii{\mathrm{(LM2)}}
\newcommand\LMiii{\mathrm{(LM3)}} % y=Xb+e
\newcommand\LMiv{\mathrm{(LM4)}}
\newcommand\LMv{\mathrm{(LM5)}}
\newcommand\LMvi{\mathrm{(LM6)}}  % Y=X beta + epsilon
\newcommand\SLRi{\mathrm{(SLR1)}}
\newcommand\SLRii{\mathrm{(SLR2)}}

\newcommand\slope{m}
\newcommand\intercept{c}

\newcommand\code[1]{\url{#1}}
\newcommand\question{{\bf Question}}
\newcommand\mysolution{{\bf Solution}}
\newcounter{Qcounter}
\newcommand\myquestion{{\stepcounter{Qcounter} \bf Question \CHAPTER.\theQcounter}}
\newcommand\myexample{{\bf Example}}
\newcommand\mydot{{\,\cdot\,}}
\newcommand\myref[1]{\m{#1}}
%\newcommand\mynotes[2]{#1}
\newcommand\mynotes[2]{#2}
\newcommand\Rspace{\mathcal{R}}

\usepackage{amsmath}
\renewcommand\vec[1]{\boldsymbol{\mathrm{#1}}}
\newcommand\vect[1]{\vec{#1}}
\newcommand\mat[1]{\mathbb{#1}}
%\newcommand\mat[1]{\mathcal{#1}}
\newcommand\mymatrix[3]{\left[
\begin{array}{cccc}
{#1}_{11} & {#1}_{12} & \dots & {#1}_{1{#3}} \\
{#1}_{21}& {#1}_{22} & \dots & {#1}_{2{#3}} \\ 
\vdots & \vdots & \ddots & \vdots \\
{#1}_{{#2}1} & {#1}_{{#2}2} & \dots & {#1}_{{#2}{#3}} 
\end{array}
\right]
}
\newcommand\myvector[2]{\left[
\begin{array}{c}
{#1}_{1} \\
{#1}_{2} \\
\vdots \\
{#1}_{{#2}}
\end{array}
\right]
}
\newcommand\mytwovector[2]{\left[
\begin{array}{c}
{#1} \\
{#2}
\end{array}
\right]
}
\newcommand\mytwomatrix[4]{\left[
\begin{array}{cc}
{#1} & {#2} \\
{#3} & {#4}
\end{array}
\right]
}
\newcommand\mysmallmatrix[3]{\left[
\begin{array}{ccc}
{#1}_{11} & \dots & {#1}_{1{#3}} \\
\vdots & \ddots & \vdots \\
{#1}_{{#2}1} & \dots & {#1}_{{#2}{#3}} 
\end{array}
\right]
}


\newcommand\bi{\begin{itemize}}
\newcommand\ei{\end{itemize}}
\newcommand\prob{\mathrm{P}}
\newcommand\E{\mathrm{E}}
\newcommand\SE{\mathrm{SE}}
\newcommand\SD{\mathrm{SD}}
\newcommand\RSS{\mathrm{RSS}}
\newcommand\SST{\mathrm{SST}}
\newcommand\pval{\mathrm{pval}}
\newcommand\var{\mathrm{Var}}
\newcommand\cov{\mathrm{Cov}}
\newcommand\given{{\, | \,}}
\newcommand\param{\,;}
\newcommand\transpose{{\raisebox{0.5mm}{\mbox{\scriptsize \textsc{t}}}}}
\newcommand\mycolon{{\hspace{0.5mm}:\hspace{0.5mm}}}
\newcommand\myemph[1]{{\textbf{#1}}}
\newcommand\mymathenv[1]{\textcolor{blue}{#1}}
\newcommand\mymath[1]{\begin{math}\textcolor{blue}{#1}\end{math}}
\newcommand\m[1]{\mymath{#1}}
\newcommand\mydisplaymath[1]{\begin{displaymath}\textcolor{blue}{#1}\end{displaymath}}
\newcommand\myeqnarray[1]{\textcolor{blue}{\begin{eqnarray*}#1 \end{eqnarray*}}}
\newcommand\myspace{\quad}
\newcommand\altdisplaymath[1]{\vspace{1mm}\textcolor{blue}{\begin{math}\displaystyle #1 \end{math}}\vspace{1mm}}
\usepackage{natbib}
\usepackage{url}
\usepackage{ulem}
\renewcommand\emph[1]{{\it #1}} % the ulem package redefines \emph
\renewcommand\em{\it} % the ulem package redefines \emph

\newcommand\enumerateSpace{\hspace{2mm}}
\usepackage{amssymb}
\newenvironment {myitemize} {
                 \begin{list}{\textcolor{black}{$\bullet$} \hfill}
%                 \begin{list}{\textcolor{blue}{{\small{$\blacktriangleright$}}} \hfill}
                 {\setlength{\labelwidth}{0.3 cm}
                  %\setlength{\leftmargin}{0em}
                  \setlength{\leftmargin}{0.15cm}
                  \setlength{\itemindent}{0.15cm}
                  \setlength{\labelsep}{0cm}
                  \setlength{\parsep}{0.2 ex}
%                  \setlength{\itemsep}{0.25 cm}
%                  \setlength{\itemsep}{0.1 cm}
                   \setlength{\itemsep}{0.0 cm}
      \setlength{\topsep}{0.0cm}}} %space between title and 1st item
   {\end{list}}

\usepackage{graphicx} % Allows including images
%\usepackage{booktabs} % Allows the use of \toprule, \midrule and \bottomrule in tables
\mode<presentation> {

\usetheme{Madrid}

\setbeamertemplate{footline} 

\setbeamertemplate{navigation symbols}{} 

}

\setlength{\parskip}{2mm}
\setlength{\parindent}{0mm}
%\newcommand\negBeforeCode{\vspace{-2mm}}
%\newcommand\negAfterCode{\vspace{-3mm}}
\newcommand\negBeforeCode{}
\newcommand\negAfterCode{}

%\renewenvironment{knitrout}{\vspace{-3mm}}{\vspace{-5mm}}





